\documentclass{article}
\usepackage{graphicx}
\usepackage{tabu}
\title{EVALUATION OF THE RESTAURANTS AROUND KATANGA, WANDEGEYA.}
\begin{document}
\begin{titlepage}
\maketitle
\begin{center}
By:
\vspace{5mm}
KIWALA MARTIN.\\
15/U/485\\
\vspace{5mm}
215000474
\end{center}
\end{titlepage}
\newpage
\begin{center}
\section{Abstract:}
\vspace{5mm}
The research was intended to search about the welfare of restaurants and their employees around katanga wandegeya.
The research was to also know about the average customers restaurants around katanga are to get in each day and to also know the moods of the employees in the different restaurants.
The research was basically done by physical interview of some employees at different restaurants and also the help of the open Data Kit to collect and upload the collected data on different servers.
\end{center}
 \begin{center}
 \section{Research Questions and Hypothesis:}
\vspace{5mm}
\begin{itemize}
\item The restaurant being researched about?
\end{itemize}
\begin{itemize}
\item The differents types of dishes served at the restaurant?
\end{itemize}
\begin{itemize}
\item The average number of customers a restaurant can get in a single day?
\end{itemize}
\begin{itemize}
\item The minimum wage paid to any employee?
\end{itemize}
The aim of the research was to answer the above questions and be able to evaluate and know the standard of restaurants around katanga inclusive of the 
welfare of the employees that work there.
\end{center}
  \begin{center}
\section{  Materials and Method:}
\vspace{5mm}
\begin{itemize}
\item ODK collect, which is a mobile phone application.
\end{itemize}
\begin{itemize}
\item The google app engine for hosting the aggregate server.
\end{itemize}
\begin{itemize}
\item The aggregate server for storing the completed forms after the different interviews.
\end{itemize}
\begin{itemize}
\item The ODK build for creating the forms used during the different interviews for the collected data.
\end{itemize}
Generally during data collection two methods were used i.e physically interviewing employees at different restaurants and also the electronic method of using the odk collect to enter the data
collected from the different interviewees and uploaded to the aggregate server for the research analysis.
Below are some screenshots of the ODK collect used during data collection:

\begin{figure}
\includegraphics[width=40mm]{odk_form.png}
\includegraphics[width=40mm]{form_sample.png}
\includegraphics[width=40mm]{interviewee.png}
\caption{Screenshots.}
\label{fig: Screenshots}
\end{figure}
\end{center}
\begin{center}
\section{  Analysis and Results:}
\vspace{5mm}
The analysis in the table below represents the different restaurants, the customers they receive each day on average and the minimum wage paid to the employees at the different restaurants.
\end{center}
TABULAR ANALYSIS:
\\
\\
\begin{tabu} to 1.0\textwidth{ | X[1] | X[c] | X[r] | }
\hline
Restaurant Name & customers @ day & Minimum wage \\
\hline
U got served&81 on average&ugx.73,000\\
\hline
Katanga joint restaurant&33 on average&ugx.45,000\\
\hline
John 3:16 restaurant&56 on average&ugx.65,000\\
\hline
Bulamu restaurant&92 on average&ugx.80,000\\
\hline
Bucks restaurant& 34 on average&ugx.100,000\\
\hline
Nnalongo end's corner join&87 on average&ugx.63,000\\
\hline
\end{tabu}
\begin{center}
\section{ Conclusions:}
 \vspace{5mm}
 Based on my research, the most liked food in katanga is mulokoni and the restaurants that have mulokoni on their menu are fairing really well in terms of average customers received per day.
Also to note that the welfare of most restaurants is well based on minimum wage given to employees which can enable them atleast attain their basic needs.
Lastly most restaurants around katanga need to improve on their cleaniliness.
\end{center}
\begin{center}
\section{ Acknowledgements:}
\vspace{5mm}
I would like to thank Mutumba Ivan in helping with my research.
I would also like to thank all the interviewees i was able to interview, thanks so much for the information provided which helped me finish this research.
\end{center}
\end{document}